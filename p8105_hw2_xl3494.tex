% Options for packages loaded elsewhere
\PassOptionsToPackage{unicode}{hyperref}
\PassOptionsToPackage{hyphens}{url}
\documentclass[
]{article}
\usepackage{xcolor}
\usepackage[margin=1in]{geometry}
\usepackage{amsmath,amssymb}
\setcounter{secnumdepth}{-\maxdimen} % remove section numbering
\usepackage{iftex}
\ifPDFTeX
  \usepackage[T1]{fontenc}
  \usepackage[utf8]{inputenc}
  \usepackage{textcomp} % provide euro and other symbols
\else % if luatex or xetex
  \usepackage{unicode-math} % this also loads fontspec
  \defaultfontfeatures{Scale=MatchLowercase}
  \defaultfontfeatures[\rmfamily]{Ligatures=TeX,Scale=1}
\fi
\usepackage{lmodern}
\ifPDFTeX\else
  % xetex/luatex font selection
\fi
% Use upquote if available, for straight quotes in verbatim environments
\IfFileExists{upquote.sty}{\usepackage{upquote}}{}
\IfFileExists{microtype.sty}{% use microtype if available
  \usepackage[]{microtype}
  \UseMicrotypeSet[protrusion]{basicmath} % disable protrusion for tt fonts
}{}
\makeatletter
\@ifundefined{KOMAClassName}{% if non-KOMA class
  \IfFileExists{parskip.sty}{%
    \usepackage{parskip}
  }{% else
    \setlength{\parindent}{0pt}
    \setlength{\parskip}{6pt plus 2pt minus 1pt}}
}{% if KOMA class
  \KOMAoptions{parskip=half}}
\makeatother
\usepackage{color}
\usepackage{fancyvrb}
\newcommand{\VerbBar}{|}
\newcommand{\VERB}{\Verb[commandchars=\\\{\}]}
\DefineVerbatimEnvironment{Highlighting}{Verbatim}{commandchars=\\\{\}}
% Add ',fontsize=\small' for more characters per line
\usepackage{framed}
\definecolor{shadecolor}{RGB}{248,248,248}
\newenvironment{Shaded}{\begin{snugshade}}{\end{snugshade}}
\newcommand{\AlertTok}[1]{\textcolor[rgb]{0.94,0.16,0.16}{#1}}
\newcommand{\AnnotationTok}[1]{\textcolor[rgb]{0.56,0.35,0.01}{\textbf{\textit{#1}}}}
\newcommand{\AttributeTok}[1]{\textcolor[rgb]{0.13,0.29,0.53}{#1}}
\newcommand{\BaseNTok}[1]{\textcolor[rgb]{0.00,0.00,0.81}{#1}}
\newcommand{\BuiltInTok}[1]{#1}
\newcommand{\CharTok}[1]{\textcolor[rgb]{0.31,0.60,0.02}{#1}}
\newcommand{\CommentTok}[1]{\textcolor[rgb]{0.56,0.35,0.01}{\textit{#1}}}
\newcommand{\CommentVarTok}[1]{\textcolor[rgb]{0.56,0.35,0.01}{\textbf{\textit{#1}}}}
\newcommand{\ConstantTok}[1]{\textcolor[rgb]{0.56,0.35,0.01}{#1}}
\newcommand{\ControlFlowTok}[1]{\textcolor[rgb]{0.13,0.29,0.53}{\textbf{#1}}}
\newcommand{\DataTypeTok}[1]{\textcolor[rgb]{0.13,0.29,0.53}{#1}}
\newcommand{\DecValTok}[1]{\textcolor[rgb]{0.00,0.00,0.81}{#1}}
\newcommand{\DocumentationTok}[1]{\textcolor[rgb]{0.56,0.35,0.01}{\textbf{\textit{#1}}}}
\newcommand{\ErrorTok}[1]{\textcolor[rgb]{0.64,0.00,0.00}{\textbf{#1}}}
\newcommand{\ExtensionTok}[1]{#1}
\newcommand{\FloatTok}[1]{\textcolor[rgb]{0.00,0.00,0.81}{#1}}
\newcommand{\FunctionTok}[1]{\textcolor[rgb]{0.13,0.29,0.53}{\textbf{#1}}}
\newcommand{\ImportTok}[1]{#1}
\newcommand{\InformationTok}[1]{\textcolor[rgb]{0.56,0.35,0.01}{\textbf{\textit{#1}}}}
\newcommand{\KeywordTok}[1]{\textcolor[rgb]{0.13,0.29,0.53}{\textbf{#1}}}
\newcommand{\NormalTok}[1]{#1}
\newcommand{\OperatorTok}[1]{\textcolor[rgb]{0.81,0.36,0.00}{\textbf{#1}}}
\newcommand{\OtherTok}[1]{\textcolor[rgb]{0.56,0.35,0.01}{#1}}
\newcommand{\PreprocessorTok}[1]{\textcolor[rgb]{0.56,0.35,0.01}{\textit{#1}}}
\newcommand{\RegionMarkerTok}[1]{#1}
\newcommand{\SpecialCharTok}[1]{\textcolor[rgb]{0.81,0.36,0.00}{\textbf{#1}}}
\newcommand{\SpecialStringTok}[1]{\textcolor[rgb]{0.31,0.60,0.02}{#1}}
\newcommand{\StringTok}[1]{\textcolor[rgb]{0.31,0.60,0.02}{#1}}
\newcommand{\VariableTok}[1]{\textcolor[rgb]{0.00,0.00,0.00}{#1}}
\newcommand{\VerbatimStringTok}[1]{\textcolor[rgb]{0.31,0.60,0.02}{#1}}
\newcommand{\WarningTok}[1]{\textcolor[rgb]{0.56,0.35,0.01}{\textbf{\textit{#1}}}}
\usepackage{graphicx}
\makeatletter
\newsavebox\pandoc@box
\newcommand*\pandocbounded[1]{% scales image to fit in text height/width
  \sbox\pandoc@box{#1}%
  \Gscale@div\@tempa{\textheight}{\dimexpr\ht\pandoc@box+\dp\pandoc@box\relax}%
  \Gscale@div\@tempb{\linewidth}{\wd\pandoc@box}%
  \ifdim\@tempb\p@<\@tempa\p@\let\@tempa\@tempb\fi% select the smaller of both
  \ifdim\@tempa\p@<\p@\scalebox{\@tempa}{\usebox\pandoc@box}%
  \else\usebox{\pandoc@box}%
  \fi%
}
% Set default figure placement to htbp
\def\fps@figure{htbp}
\makeatother
\setlength{\emergencystretch}{3em} % prevent overfull lines
\providecommand{\tightlist}{%
  \setlength{\itemsep}{0pt}\setlength{\parskip}{0pt}}
\usepackage{bookmark}
\IfFileExists{xurl.sty}{\usepackage{xurl}}{} % add URL line breaks if available
\urlstyle{same}
\hypersetup{
  pdftitle={p8105\_hw2\_xl3494},
  hidelinks,
  pdfcreator={LaTeX via pandoc}}

\title{p8105\_hw2\_xl3494}
\author{}
\date{\vspace{-2.5em}}

\begin{document}
\maketitle

\begin{Shaded}
\begin{Highlighting}[]
\FunctionTok{library}\NormalTok{(tidyverse)}
\end{Highlighting}
\end{Shaded}

\begin{verbatim}
## -- Attaching core tidyverse packages ------------------------ tidyverse 2.0.0 --
## v dplyr     1.1.4     v readr     2.1.5
## v forcats   1.0.0     v stringr   1.5.1
## v ggplot2   3.5.2     v tibble    3.3.0
## v lubridate 1.9.4     v tidyr     1.3.1
## v purrr     1.1.0     
## -- Conflicts ------------------------------------------ tidyverse_conflicts() --
## x dplyr::filter() masks stats::filter()
## x dplyr::lag()    masks stats::lag()
## i Use the conflicted package (<http://conflicted.r-lib.org/>) to force all conflicts to become errors
\end{verbatim}

\begin{Shaded}
\begin{Highlighting}[]
\FunctionTok{library}\NormalTok{(lubridate)}
\FunctionTok{library}\NormalTok{(readxl)}
\end{Highlighting}
\end{Shaded}

\section{Problem 1}\label{problem-1}

Clean the data in pols-month.csv:

\begin{Shaded}
\begin{Highlighting}[]
\NormalTok{pols\_month }\OtherTok{=}
  \FunctionTok{read\_csv}\NormalTok{(}\StringTok{"./fivethirtyeight\_datasets/pols{-}month.csv"}\NormalTok{)}\SpecialCharTok{|\textgreater{}}
  \FunctionTok{separate}\NormalTok{(mon, }\AttributeTok{into =} \FunctionTok{c}\NormalTok{(}\StringTok{"year"}\NormalTok{, }\StringTok{"nmonth"}\NormalTok{, }\StringTok{"day"}\NormalTok{), }\AttributeTok{sep =} \StringTok{"{-}"}\NormalTok{, }\AttributeTok{convert =} \ConstantTok{TRUE}\NormalTok{) }\SpecialCharTok{|\textgreater{}}
  \FunctionTok{mutate}\NormalTok{(}
    \AttributeTok{month =} \FunctionTok{factor}\NormalTok{(}\FunctionTok{tolower}\NormalTok{(month.name[nmonth]),}
                   \AttributeTok{levels =} \FunctionTok{tolower}\NormalTok{(month.name),}
                   \AttributeTok{ordered =} \ConstantTok{TRUE}\NormalTok{)) }\SpecialCharTok{|\textgreater{}}
  \FunctionTok{mutate}\NormalTok{(}\AttributeTok{president =} \FunctionTok{if\_else}\NormalTok{(prez\_gop }\SpecialCharTok{==} \DecValTok{1}\NormalTok{, }\StringTok{"gop"}\NormalTok{, }\StringTok{"dem"}\NormalTok{)) }\SpecialCharTok{|\textgreater{}}
  \FunctionTok{arrange}\NormalTok{(year, month) }\SpecialCharTok{|\textgreater{}}
  \FunctionTok{select}\NormalTok{(year, month, }\FunctionTok{everything}\NormalTok{(), }\SpecialCharTok{{-}}\NormalTok{nmonth, }\SpecialCharTok{{-}}\NormalTok{day, }\SpecialCharTok{{-}}\NormalTok{prez\_gop, }\SpecialCharTok{{-}}\NormalTok{prez\_dem)}
\end{Highlighting}
\end{Shaded}

\begin{verbatim}
## Rows: 822 Columns: 9
## -- Column specification --------------------------------------------------------
## Delimiter: ","
## dbl  (8): prez_gop, gov_gop, sen_gop, rep_gop, prez_dem, gov_dem, sen_dem, r...
## date (1): mon
## 
## i Use `spec()` to retrieve the full column specification for this data.
## i Specify the column types or set `show_col_types = FALSE` to quiet this message.
\end{verbatim}

Clean the data in snp.csv:

\begin{Shaded}
\begin{Highlighting}[]
\NormalTok{snp }\OtherTok{=}
  \FunctionTok{read\_csv}\NormalTok{(}\StringTok{"./fivethirtyeight\_datasets/snp.csv"}\NormalTok{)}\SpecialCharTok{|\textgreater{}}
  \FunctionTok{separate}\NormalTok{(date, }\AttributeTok{into =} \FunctionTok{c}\NormalTok{(}\StringTok{"m"}\NormalTok{, }\StringTok{"d"}\NormalTok{, }\StringTok{"yy"}\NormalTok{), }\AttributeTok{sep =} \StringTok{"/"}\NormalTok{, }\AttributeTok{convert =} \ConstantTok{TRUE}\NormalTok{) }\SpecialCharTok{|\textgreater{}}
  \FunctionTok{mutate}\NormalTok{(}
    \AttributeTok{year  =} \FunctionTok{if\_else}\NormalTok{(yy }\SpecialCharTok{\textgreater{}=} \DecValTok{50}\NormalTok{, }\DecValTok{1900}\DataTypeTok{L} \SpecialCharTok{+}\NormalTok{ yy, }\DecValTok{2000}\DataTypeTok{L} \SpecialCharTok{+}\NormalTok{ yy),   }\CommentTok{\# 50–99 → 1950–1999;00–49 → 2000–2049}
    \AttributeTok{month =} \FunctionTok{factor}\NormalTok{(}\FunctionTok{tolower}\NormalTok{(month.name[m]),}
                   \AttributeTok{levels =} \FunctionTok{tolower}\NormalTok{(month.name), }\AttributeTok{ordered =} \ConstantTok{TRUE}\NormalTok{)}
\NormalTok{  ) }\SpecialCharTok{|\textgreater{}}
  \FunctionTok{arrange}\NormalTok{(year, month) }\SpecialCharTok{|\textgreater{}}
  \FunctionTok{select}\NormalTok{(year, month, close)}
\end{Highlighting}
\end{Shaded}

\begin{verbatim}
## Rows: 787 Columns: 2
## -- Column specification --------------------------------------------------------
## Delimiter: ","
## chr (1): date
## dbl (1): close
## 
## i Use `spec()` to retrieve the full column specification for this data.
## i Specify the column types or set `show_col_types = FALSE` to quiet this message.
\end{verbatim}

Tidy the unemployment data:

\begin{Shaded}
\begin{Highlighting}[]
\NormalTok{unemployment }\OtherTok{=}
  \FunctionTok{read\_csv}\NormalTok{(}\StringTok{"./fivethirtyeight\_datasets/unemployment.csv"}\NormalTok{)}\SpecialCharTok{|\textgreater{}}
\NormalTok{  janitor}\SpecialCharTok{::}\FunctionTok{clean\_names}\NormalTok{() }\SpecialCharTok{|\textgreater{}}
  \FunctionTok{pivot\_longer}\NormalTok{(}
    \AttributeTok{cols =} \SpecialCharTok{{-}}\NormalTok{year,}
    \AttributeTok{names\_to  =} \StringTok{"ab\_month"}\NormalTok{,}
    \AttributeTok{values\_to =} \StringTok{"unemployment"}\NormalTok{) }\SpecialCharTok{|\textgreater{}}
  \FunctionTok{mutate}\NormalTok{(}
    \AttributeTok{m\_idx =} \FunctionTok{match}\NormalTok{(ab\_month, }\FunctionTok{tolower}\NormalTok{(month.abb)),             }
    \AttributeTok{month =} \FunctionTok{factor}\NormalTok{(}\FunctionTok{tolower}\NormalTok{(month.name[m\_idx]),}
                   \AttributeTok{levels =} \FunctionTok{tolower}\NormalTok{(month.name), }\AttributeTok{ordered =} \ConstantTok{TRUE}\NormalTok{)) }\SpecialCharTok{|\textgreater{}}
  \FunctionTok{arrange}\NormalTok{(year, month) }\SpecialCharTok{|\textgreater{}}
  \FunctionTok{select}\NormalTok{(year, month, unemployment)}
\end{Highlighting}
\end{Shaded}

\begin{verbatim}
## Rows: 68 Columns: 13
## -- Column specification --------------------------------------------------------
## Delimiter: ","
## dbl (13): Year, Jan, Feb, Mar, Apr, May, Jun, Jul, Aug, Sep, Oct, Nov, Dec
## 
## i Use `spec()` to retrieve the full column specification for this data.
## i Specify the column types or set `show_col_types = FALSE` to quiet this message.
\end{verbatim}

Joining snp into pols\_month:

\begin{Shaded}
\begin{Highlighting}[]
\NormalTok{pols\_snp }\OtherTok{\textless{}{-}}\NormalTok{ pols\_month }\SpecialCharTok{|\textgreater{}}
  \FunctionTok{left\_join}\NormalTok{(snp }\SpecialCharTok{|\textgreater{}} \FunctionTok{rename}\NormalTok{(}\AttributeTok{snp\_close =}\NormalTok{ close),}
            \AttributeTok{by =} \FunctionTok{c}\NormalTok{(}\StringTok{"year"}\NormalTok{,}\StringTok{"month"}\NormalTok{))}
\end{Highlighting}
\end{Shaded}

Joining unemployment into result:

\begin{Shaded}
\begin{Highlighting}[]
\NormalTok{pols\_snp\_unemp }\OtherTok{\textless{}{-}}\NormalTok{ pols\_snp }\SpecialCharTok{|\textgreater{}}
  \FunctionTok{left\_join}\NormalTok{(unemployment, }\AttributeTok{by =} \FunctionTok{c}\NormalTok{(}\StringTok{"year"}\NormalTok{,}\StringTok{"month"}\NormalTok{))}
\end{Highlighting}
\end{Shaded}

\subsubsection{About the datasets}\label{about-the-datasets}

The pols-month table contains the monthly counts of politicians by party
-- Republican/Democratic governors, senators, and representatives, and
the party of the sitting president. After cleaning, the time range is
from Jan 1947 to Jun 2015 (822 rows).

The snp table holds the Standard \& Poor's stock market index (close) by
year and month. After cleaning, it contains 787 rows with a time range
from Jan 1950 to Jul 2015.

The unemployment table records the monthly unemployment rate (\%); we
pivoted it from wide (Jan--Dec columns) to long with the same keys --
year and month. The dataset contains 816 rows with a time range from Jan
1948 to Jun 2015.

The joined dataset pols\_snp\_unemp contains 822 rows and 11 variables.
Key variables include year, month, president, party counts (gov\_gop,
sen\_gop, rep\_gop, gov\_dem, sen\_dem, and rep\_dem), stock market
index (snp\_close), and \% unemployment. The final table keeps all the
pol-month months from Jan 1947 to Jun 2015. The coverage of the other
two dataset differs slightly: snp\_close covers from Jan 1950 to Jun
2015, and unemployment covers from Jan 1948 to Jun 2015.

\section{Problem 2}\label{problem-2}

Read in and clean the Mr.~Trash Wheel sheet:

\begin{Shaded}
\begin{Highlighting}[]
\NormalTok{mr }\OtherTok{=} 
  \FunctionTok{read\_excel}\NormalTok{(}\StringTok{"./trashwheel\_data/202509 Trash Wheel Collection Data.xlsx"}\NormalTok{,}
                           \AttributeTok{sheet =} \StringTok{"Mr. Trash Wheel"}\NormalTok{,}
                           \AttributeTok{range =} \StringTok{"A2:N709"}\NormalTok{) }\SpecialCharTok{|\textgreater{}}
\NormalTok{  janitor}\SpecialCharTok{::}\FunctionTok{clean\_names}\NormalTok{()}\SpecialCharTok{|\textgreater{}}
  \FunctionTok{filter}\NormalTok{(}\SpecialCharTok{!}\FunctionTok{is.na}\NormalTok{(dumpster)) }\SpecialCharTok{|\textgreater{}}
  \FunctionTok{mutate}\NormalTok{(}\AttributeTok{month =} \FunctionTok{tolower}\NormalTok{(month),}
         \AttributeTok{year  =} \FunctionTok{as.integer}\NormalTok{(readr}\SpecialCharTok{::}\FunctionTok{parse\_number}\NormalTok{(year)),}
         \AttributeTok{date  =} \FunctionTok{as.Date}\NormalTok{(date),}
         \AttributeTok{sports\_balls =} \FunctionTok{as.integer}\NormalTok{(}\FunctionTok{round}\NormalTok{(sports\_balls)),}
         \AttributeTok{wheel =} \StringTok{"mr"}\NormalTok{)}
\end{Highlighting}
\end{Shaded}

Read in and clean the Professor Trash Wheel sheet:

\begin{Shaded}
\begin{Highlighting}[]
\NormalTok{professor }\OtherTok{=} 
  \FunctionTok{read\_excel}\NormalTok{(}\StringTok{"./trashwheel\_data/202509 Trash Wheel Collection Data.xlsx"}\NormalTok{,}
                           \AttributeTok{sheet =} \StringTok{"Professor Trash Wheel"}\NormalTok{,}
                           \AttributeTok{range =} \StringTok{"A2:M134"}\NormalTok{) }\SpecialCharTok{|\textgreater{}}
\NormalTok{  janitor}\SpecialCharTok{::}\FunctionTok{clean\_names}\NormalTok{()}\SpecialCharTok{|\textgreater{}}
  \FunctionTok{filter}\NormalTok{(}\SpecialCharTok{!}\FunctionTok{is.na}\NormalTok{(dumpster)) }\SpecialCharTok{|\textgreater{}}
  \FunctionTok{mutate}\NormalTok{(}\AttributeTok{month =} \FunctionTok{tolower}\NormalTok{(month),}
         \AttributeTok{date  =} \FunctionTok{as.Date}\NormalTok{(date),}
         \AttributeTok{wheel =} \StringTok{"professor"}\NormalTok{)}
\end{Highlighting}
\end{Shaded}

Read in and clean the Gwynns Falls Trash Wheel sheet:

\begin{Shaded}
\begin{Highlighting}[]
\NormalTok{gwynns }\OtherTok{=} 
  \FunctionTok{read\_excel}\NormalTok{(}\StringTok{"./trashwheel\_data/202509 Trash Wheel Collection Data.xlsx"}\NormalTok{,}
                           \AttributeTok{sheet =} \StringTok{"Gwynns Falls Trash Wheel"}\NormalTok{,}
                           \AttributeTok{range =} \StringTok{"A2:L351"}\NormalTok{) }\SpecialCharTok{|\textgreater{}}
\NormalTok{  janitor}\SpecialCharTok{::}\FunctionTok{clean\_names}\NormalTok{()}\SpecialCharTok{|\textgreater{}}
  \FunctionTok{filter}\NormalTok{(}\SpecialCharTok{!}\FunctionTok{is.na}\NormalTok{(dumpster)) }\SpecialCharTok{|\textgreater{}}
  \FunctionTok{mutate}\NormalTok{(}\AttributeTok{month =} \FunctionTok{tolower}\NormalTok{(month),}
         \AttributeTok{date  =} \FunctionTok{as.Date}\NormalTok{(date),}
         \AttributeTok{wheel =} \StringTok{"gwynns"}\NormalTok{)}
\end{Highlighting}
\end{Shaded}

Bind all rows:

\begin{Shaded}
\begin{Highlighting}[]
\NormalTok{wheel\_bind }\OtherTok{=} 
  \FunctionTok{bind\_rows}\NormalTok{(mr, professor, gwynns) }\SpecialCharTok{|\textgreater{}}
  \FunctionTok{select}\NormalTok{(wheel, dumpster, month, year, date, }\FunctionTok{everything}\NormalTok{())}
\end{Highlighting}
\end{Shaded}

\subsubsection{About the dataset}\label{about-the-dataset}

The combined Trash Wheel dataset contains 1188 observations and 15
variables spanning 2014--2025. Each row contains a dumpster with
identifiers \texttt{dumpster} and \texttt{wheel} (Mr., Professor,
Gwynnda), timestamp \texttt{date} (plus \texttt{year}, \texttt{month}),
and material metrics such as \texttt{weight\_tons},
\texttt{volume\_cubic\_yards}, \texttt{plastic\_bottles},
\texttt{polystyrene}, \texttt{cigarette\_butts}, \texttt{wrappers},
\texttt{sports\_balls} (rounded to integers), and
\texttt{homes\_powered}.

Professor Trash Wheel collected a total of 282.26 tons of trash.

In June 2022, Gwynnda collected 18,120 cigarette butts.

\section{Problem 3}\label{problem-3}

read in and clean zillow\_data

\begin{Shaded}
\begin{Highlighting}[]
\NormalTok{zori\_data }\OtherTok{=} 
  \FunctionTok{read\_csv}\NormalTok{(}\StringTok{"./zillow\_data/Zip\_zori\_uc\_sfrcondomfr\_sm\_month\_NYC.csv"}\NormalTok{) }\SpecialCharTok{|\textgreater{}}
\NormalTok{  janitor}\SpecialCharTok{::}\FunctionTok{clean\_names}\NormalTok{() }\SpecialCharTok{|\textgreater{}}
  \FunctionTok{select}\NormalTok{ (}
\NormalTok{    region\_id,}
    \AttributeTok{county =}\NormalTok{ county\_name,}
    \AttributeTok{zip\_code =}\NormalTok{ region\_name,}
    \FunctionTok{everything}\NormalTok{()) }\SpecialCharTok{|\textgreater{}}
  \FunctionTok{select}\NormalTok{ (}\SpecialCharTok{{-}}\NormalTok{region\_type, }\SpecialCharTok{{-}}\NormalTok{state\_name, }\SpecialCharTok{{-}}\NormalTok{state, }\SpecialCharTok{{-}}\NormalTok{city, }\SpecialCharTok{{-}}\NormalTok{metro)}\SpecialCharTok{|\textgreater{}}
  \FunctionTok{pivot\_longer}\NormalTok{(}\AttributeTok{cols =} \FunctionTok{matches}\NormalTok{(}\StringTok{"\^{}x"}\NormalTok{),}
               \AttributeTok{names\_to =} \StringTok{"date\_raw"}\NormalTok{,}
               \AttributeTok{values\_to =} \StringTok{"zori"}\NormalTok{) }\SpecialCharTok{|\textgreater{}}
  \FunctionTok{mutate}\NormalTok{(}
    \AttributeTok{date =}\NormalTok{ date\_raw }\SpecialCharTok{|\textgreater{}}
      \FunctionTok{str\_remove}\NormalTok{(}\StringTok{"\^{}x"}\NormalTok{) }\SpecialCharTok{|\textgreater{}}
      \FunctionTok{str\_replace\_all}\NormalTok{(}\StringTok{"\_"}\NormalTok{, }\StringTok{"{-}"}\NormalTok{) }\SpecialCharTok{|\textgreater{}}
      \FunctionTok{as.Date}\NormalTok{()) }\SpecialCharTok{|\textgreater{}}
  \FunctionTok{mutate}\NormalTok{(}
    \AttributeTok{county =}\NormalTok{ county }\SpecialCharTok{|\textgreater{}}
      \FunctionTok{str\_remove}\NormalTok{(}\FunctionTok{regex}\NormalTok{(}\StringTok{"}\SpecialCharTok{\textbackslash{}\textbackslash{}}\StringTok{s*county}\SpecialCharTok{\textbackslash{}\textbackslash{}}\StringTok{b"}\NormalTok{, }\AttributeTok{ignore\_case =} \ConstantTok{TRUE}\NormalTok{)) }\SpecialCharTok{|\textgreater{}}
      \FunctionTok{str\_squish}\NormalTok{()) }\SpecialCharTok{|\textgreater{}}
  \FunctionTok{select}\NormalTok{(region\_id, county, zip\_code, size\_rank, date, zori)}
\end{Highlighting}
\end{Shaded}

\begin{verbatim}
## Rows: 149 Columns: 125
## -- Column specification --------------------------------------------------------
## Delimiter: ","
## chr   (6): RegionType, StateName, State, City, Metro, CountyName
## dbl (119): RegionID, SizeRank, RegionName, 2015-01-31, 2015-02-28, 2015-03-3...
## 
## i Use `spec()` to retrieve the full column specification for this data.
## i Specify the column types or set `show_col_types = FALSE` to quiet this message.
\end{verbatim}

Read in and tidy zip\_code\_data

\begin{Shaded}
\begin{Highlighting}[]
\NormalTok{zip\_code\_data }\OtherTok{=} 
  \FunctionTok{read\_csv}\NormalTok{(}\StringTok{"./zillow\_data/Zip Codes.csv"}\NormalTok{) }\SpecialCharTok{|\textgreater{}}
\NormalTok{  janitor}\SpecialCharTok{::}\FunctionTok{clean\_names}\NormalTok{() }\SpecialCharTok{|\textgreater{}}
  \FunctionTok{select}\NormalTok{(county, zip\_code, neighborhood, county\_fips)}
\end{Highlighting}
\end{Shaded}

\begin{verbatim}
## Rows: 322 Columns: 7
## -- Column specification --------------------------------------------------------
## Delimiter: ","
## chr (4): County, County Code, File Date, Neighborhood
## dbl (3): State FIPS, County FIPS, ZipCode
## 
## i Use `spec()` to retrieve the full column specification for this data.
## i Specify the column types or set `show_col_types = FALSE` to quiet this message.
\end{verbatim}

Join two datasets

\begin{Shaded}
\begin{Highlighting}[]
\NormalTok{zori\_final }\OtherTok{=} 
   \FunctionTok{left\_join}\NormalTok{(zip\_code\_data, zori\_data, }
             \AttributeTok{by =} \FunctionTok{c}\NormalTok{(}\StringTok{"zip\_code"}\NormalTok{,}\StringTok{"county"}\NormalTok{), }
             \AttributeTok{relationship =} \StringTok{"many{-}to{-}many"}\NormalTok{) }\SpecialCharTok{|\textgreater{}}
  \FunctionTok{filter}\NormalTok{(}\SpecialCharTok{!}\FunctionTok{is.na}\NormalTok{(zori)) }\SpecialCharTok{|\textgreater{}}
  \FunctionTok{pivot\_wider}\NormalTok{(}
    \AttributeTok{names\_from =} \StringTok{"date"}\NormalTok{,}
    \AttributeTok{values\_from =} \StringTok{"zori"}\NormalTok{)}

\NormalTok{zori\_final}
\end{Highlighting}
\end{Shaded}

\begin{verbatim}
## # A tibble: 148 x 122
##    county zip_code neighborhood     county_fips region_id size_rank `2023-01-31`
##    <chr>     <dbl> <chr>                  <dbl>     <dbl>     <dbl>        <dbl>
##  1 Bronx     10451 High Bridge and~       36005     61791       838        2711.
##  2 Bronx     10452 High Bridge and~       36005     61792       112          NA 
##  3 Bronx     10453 Central Bronx          36005     61793        84          NA 
##  4 Bronx     10454 Hunts Point and~       36005     61794      2018        2624.
##  5 Bronx     10455 Hunts Point and~       36005     61795      1733          NA 
##  6 Bronx     10456 High Bridge and~       36005     61796        49          NA 
##  7 Bronx     10457 Central Bronx          36005     61797       107        2152.
##  8 Bronx     10458 Bronx Park and ~       36005     61798        63        1585.
##  9 Bronx     10459 Hunts Point and~       36005     61799       914          NA 
## 10 Bronx     10460 Central Bronx          36005     61800       462          NA 
## # i 138 more rows
## # i 115 more variables: `2023-02-28` <dbl>, `2023-03-31` <dbl>,
## #   `2023-04-30` <dbl>, `2023-05-31` <dbl>, `2023-06-30` <dbl>,
## #   `2023-07-31` <dbl>, `2023-08-31` <dbl>, `2023-09-30` <dbl>,
## #   `2023-10-31` <dbl>, `2023-11-30` <dbl>, `2023-12-31` <dbl>,
## #   `2024-01-31` <dbl>, `2024-02-29` <dbl>, `2024-03-31` <dbl>,
## #   `2024-04-30` <dbl>, `2024-05-31` <dbl>, `2024-06-30` <dbl>, ...
\end{verbatim}

\end{document}
